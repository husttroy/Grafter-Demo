\section{Grafter Implementation}
{\grafter} takes a pair of clones and an existing test suite as input and constrasts the runtime behavior of the clones. The core of {\grafter} is to transplant test cases between clones so that {\grafter} can exercise clones with the same test. This section briefly discusses the transplantation process of {\grafter}. The details are described in our previous work~\cite{zhang2017automated}.

\noindent{\bf Variation identification.} {\grafter} first perform inter-procedural analysis to identify the input and output parameters referenced by each clone and its subroutines to expose its de-facto testing interface. {\grafter} then matches the input and output parameters between clones and detect the paramaters that are defined in one clone but not in the counterpart clone.

\noindent{\bf Code transplantation.} Simply copying and pasting a clone in place of its counterpart clone could lead to compilation errors due to variations in clone content. {\grafter} applies five transplantation
rules to ensure type safety during code transplantation.  

\noindent{\bf Data propagation.} Similar to how surgeons reattach blood vessels to ensure the
blood in the recipient flows correctly to the vessels of the transplanted organ, {\grafter} also needs to make sure the input data flows correctly into the grafted clone and the updated values flows back to the same assertion check of the original test. Therefore, {\grafter} applies four heuristics to insert stub code to
ensure that (1) newly declared variables consume the same input data as their counterparts in the recipient and (2) the updated values flow back to the same test oracle.

\noindent{\bf Differential testing.} {\grafter} supports behavior comparison at two levels. The test-level comparison runs the same test on two clones and compares the test outcomes. If a test succeeds on one clone but fails on the other, behavior divergence is noted. The state-level comparison compares the intermediate program states for affected variables at the exit(s) of the clones. {\grafter} instruments code clones to capture the updated program states at the exit(s) of clones. {\grafter} uses the XStream library\footnote{\url{http://x-stream.github.io/}} to serialize the program states of affected variables in an XML format. Then it checks if two clones update corresponding variables with the same values.

{\grafter} is built on top of an open-source visual diff and merge tool, JMeld.\footnote{\url{https://github.com/albfan/jmeld}} Configure {\grafter} and run it using the following command. The user configuration file asks the user to speficy the source and test folders in a project as well as the build system in the command so that {\grafter} can perform the test coverage analysis.

\begin{lstlisting}
$ java -jar grafter.jar /path/to/your/config/file
\end{lstlisting}